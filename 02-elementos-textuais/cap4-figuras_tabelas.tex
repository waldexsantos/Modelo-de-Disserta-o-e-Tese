
\chapter{Figuras, Tabelas e outros elementos}\label{chap:fundamentacaoTeorica}
A seguir ilustra-se a forma de incluir figuras, tabelas, equações, siglas e símbolos no documento, obtendo indexação automática em suas respectivas listas.
A numeração sequencial de figuras, tabelas e equações ocorre de modo automático.
Referências cruzadas são obtidas através dos comandos \verb#\label{}# e \verb#\ref{}#.
Por exemplo, estou me referindo agora a introdução que corresponde ao  capítulo \ref{chap:introducao}. Todos os comandos utilizados para gerar os elementos desse texto estão nos respectivos arquivos dos capítulos. Não os colocaremos aqui para que o texto não fique demasiadamente extenso e não ficarmos repetitivos.

\section{Figuras}
\label{sec:figuras}

Abaixo é apresentado um exemplo de figura. A  figura \ref{fig:brasaouesc} aparece automaticamente na lista de figuras. Para uso avançado de imagens no LATEX, recomenda-se a consulta de literatura especializada.

\begin{figure}[!htb]
\centering
	\includegraphics[width=0.3\textwidth]{./04-figuras/brasaouesc}\label{fig:brasaouesc}
    \caption{Brasão da UESC}\vspace{-0.4cm}
    \fonte{\citeonline{teste:2014}}

\end{figure}

\subsection{Figuras lado a lado}\label{figladoalado}
Para colocar figuras lado a lado com legendas diferentes utilizamos o pacote {\ttfamily subfigure}, declarado no preâmbulo. Porém o pacote abntex, no qual nos baseamos, é incompatível com o pacote {\ttfamily subfigure}. Ao tentar usá-lo a compilação apresenta erros de incompatibilidade. 

Para contornar essa situação fiz algumas alterações necessárias no pacote e o renomeei como {\ttfamily subfigureppmgc}. Assim, caso você queira utilizar no seu texto figuras lado a lado com legendas diferentes, deve utilizar essa nova versão do  pacote, o qual se encontra na pasta {\ttfamily pacotes}. O template já está configurada para utilizá-lo automaticamente.

Caso as figuras que fiquem lado a lado  utilizem apenas uma legenda não há necessidade do pacote, basta chamá-las normalmente no ambiente {\ttfamily figure}.

Abaixo segue um exemplo de como fica a utilização do pacote inserindo figuras lado a lado, com legendas diferentes.\\

\begin{figure}[!htb]
 \subfigure[\label{fig:rc-adezanoss}]{
\includegraphics[width=7cm,height=6cm]{04-figuras/rc-adezanos}}
\subfigure[\label{fig:rc-atualmentee}]{
\includegraphics[width=7cm,height=6cm]{04-figuras/rc-atualmente}}
\caption{Imagens comparando  o rio Cachoeira a dez anos atrás (a)  e em 2013 (b)} \label{fig:estadorc}\vspace{-0.6cm}
\fonte{Pimenta Blog.BR, no endereço em $<$http://www.pimenta.blog.br/$>$}
\end{figure}

Este \href{https://www.overleaf.com/learn/latex/Inserting_Images}{link do overleaf} contém mais informações de como trabalhar com figuras em latex. 
\section{Quadros e Tabelas}
\label{sec:tabelas}

Também é apresentado o exemplo do quadro \ref{qua:comparabd} e da tabela \ref{tab:correlacao}, que aparece automaticamente na lista de quadros e tabelas.
Informações sobre a construção de tabelas no LATEX podem ser encontradas na literatura especializada e neste link de ajuda do overleaf: \href{https://www.overleaf.com/learn/latex/Tables}{https://www.overleaf.com/learn/latex/Tables}.

\input{./06-quadros/quacomparabd}
\newpage
\textbf{Exemplos de tabelas:
}
\input{./05-tabelas/tabcorrelacao}

\input{./05-tabelas/tabteste}

\section{Equações}
\label{sec:equacoes}

A transformada de Laplace é dada na equação \ref{eq:laplace}, enquanto a equação \ref{eq:dft} apresenta a formulação da transformada discreta de Fourier bidimensional\footnote{Deve-se reparar na formatação esteticamente perfeita destas equações.}.

\begin{equation}
	X(s) = \int\limits_{t = -\infty}^{\infty} x(t) \, \text{e}^{-st} \, dt
	\label{eq:laplace}
\end{equation}

\begin{equation}
	F(u, v) = \sum_{m = 0}^{M - 1} \sum_{n = 0}^{N - 1} f(m, n) \exp \left[ -j 2 \pi \left( \frac{u m}{M} + \frac{v n}{N} \right) \right]
	\label{eq:dft}
\end{equation}

\section{Algoritmos}
Ja declaramos no preâmbulo o pacote {\ttfamily algorithm2e} necessário para utilizar algoritmos em latex. Isto é feito dentro do ambiente {\ttfamily algorithm}, conforme exemplos  disponíveis na pasta {\ttfamily 07-algoritmos}. Segue abaixo um primeiro exemplo:\\
\begin{algorithm}[H]
\caption{Como escrever algoritmos  em \LaTeX2e\label{meualgoritmo}}
	\KwData{$\Delta i, S$}
	\KwResult{N}
		\While{$\Delta i > N$}{
			imprima()\;
			$\Delta i \leftarrow N^2$ \;
			\eIf{b}{
				atualiza os Valores\;
				$N \leftarrow V_f + 2$\;
			}{
				$N \leftarrow S_i - 7$\;
			}
			
			$N \leftarrow N + 1$ \;
		}
\end{algorithm}
Referenciando-o no texto \autoref{meualgoritmo}.

No exemplo acima, utilizamos \verb#\KwData{}#  para declara as variáveis de entrada (pode-se utilizar também \verb#\KwIn{}#) e  \verb#\KwResult{}# para variáveis de saída (ou poderia ser utilizado \verb#\KwOut{}#. 

\subsection{Comandos para escrita dos algoritmos}
Os comandos que podem ser utilizados estão na documentação do pacote {\ttfamily algorithm2e}, mas segue abaixo um pequeno resumo de alguns dos mais úteis:
\begin{enumerate}
    \item Ponto e Vírgula:
    \begin{itemize}
        \item \verb#\;#
    \end{itemize}
    \item Ifs:
    \begin{itemize}
        \item \verb#\If{condição}{then block#
        \item \verb#\Else{else block}#
\item \verb#\eIf{condição}{then block}{else block}#
    \end{itemize}
    \item Switchs:
    \begin{itemize}
        \item \verb#\\Switch{condição}{Switch block}#
        \item \verb#\\Case{a case}{case block}#
        \item \verb#\\Other{otherwise block}#
    \end{itemize}
    \item Loops:
    \begin{itemize}
        \item \verb#\For{condition}{text loop}#
    \item \verb#\While{condition}{text loop}#
    \item \verb#\ForEach{condition}{text loop}#
    \item \verb#\Repeat{end condition}{text loop}#
    \end{itemize}
    \item Input e Output:
    \begin{itemize}
        \item \verb#\KwIn{input}#
\item \verb#\KwOut{output}#
    \end{itemize}
    \item Comentários:
    \begin{itemize}
        \item \verb#\tcc{line(s) of comment}: Comentários de Trecho (/* */)#
    \item \verb#\tcp{line(s) of comment}: Comentários de Linha (// )#
    \end{itemize}
\end{enumerate}



Abaixo apresentamos mais um exemplo em que há apenas \verb#if# e não há \verb#else#. Neste caso utilizamos o comando: \verb#\If{condição}{se a condição for verdadeira}}#\\
\input{07-algoritmos/algoritmo2}
