
\chapter{Chamada dos arquivos dentro do código principal}
O arquivo principal a ser copilado é {\ttfamily PPGMC.tex}. Nele são chamados os  arquivos/elementos da sua dissertação (ou tese). Faça as modificações após o comando \verb#\begin{document}#. Depois abra cada um dos documentos que estão sendo chamados pelo documento mestre e modifique-os para obter o seu documento final. Todos os arquivos dos elementos pré-textuais, textuais e pós-textuais devem ser editados pelos autores, segundo suas necessidades.

\section{Arquivos, Pacotes e configurações}
O arquivo {\ttfamily pacoteprincipal.tex}, chamado no arquivo principal com o comando \verb#\documentclass[oneside]{pacotes/ppgmc-uesc} 	% imprimir apenas frente
%\documentclass[doubleside]{pacotes/ppgmc-uesc}	% imprimir frente e verso

% importações de pacotes
	\usepackage[alf, abnt-emphasize=bf, bibjustif, recuo=0cm, abnt-etal-cite=2, abnt-etal-list=0]{abntex2cite}	% citações padrão ABNT
	\usepackage{bookmark}					% cria menu de bookmarks
	\usepackage[utf8]{inputenc}				% acentuação direta
	\usepackage[T1]{fontenc}				% codificação da fonte em 8 bits
	\usepackage{graphicx}					% inserir figuras
	\usepackage{float}				    	%PERMITE usar H em figura
	%\usepackage{picinpar}					% imagens ao lado do texto
	\usepackage{amsfonts, amssymb, amsmath}		% fonte e simbolos matemáticos
	\usepackage{cancel}                     %cancelar termos numa expressão matemática
	\usepackage{lastpage}			% Usado pela Ficha catalográfica
	\usepackage{verbatim}					% texto é interpretado como escrito no documento
	\usepackage{multirow, array}				% múltiplas linhas e colunas em     tabelas
	\usepackage{indentfirst}				% indenta o primeiro parágrafo de cada seção.
	\usepackage[algoruled, portuguese]{algorithm2e}		% escrever algoritmos

	%\usepackage{times}				    	% usa a fonte Times
	\usepackage{hyperref} 		            % inserindo links azuis, referências em preto
	\usepackage{pacotes/subfigureppgmc}					% posicionamento de figuras
	\usepackage{multicol}                   % para usar o ambiente multicols-multiplas colunas
%	\usepackage{paralist}                  %permite listas especiais
	\usepackage{xcolor, colortbl}			% comandos de cores
	%\usepackage{breakurl}					% permite quebra de linha em urls
	
	%\usepackage{subeqnarray}				% sub enumeração de equações
	%\usepackage{makeidx}					% produzir índice remissivo (glossario)
	%\usepackage{multind}					% produzir índices múltiplos
	\usepackage{pdfpages}               % inserir arq. pdf no texto (ex.: folha de aprovação e ficha catalográfica)
	\usepackage{pdflscape}                % permite colocar página em formato paisagem
%	\usepackage[normalem]{ulem} % para o underline colorido

\makeatother
#, contém os principais pacotes utilizados. Já o comando \verb#% define as cores dos links e informações do pdf
\makeatletter
\hypersetup{
	portuguese,
	colorlinks,
	linkcolor=blue,
	citecolor=blue,
	filecolor=blue,
	urlcolor=blue,
	breaklinks=true,
	pdftitle={\@title},
	pdfauthor={\@author},
	pdfsubject={\imprimirpreambulo},
	pdfkeywords={abnt, latex, abntex, abntex2}
}}# chama o arquivo de configurações de cores. Ambos os arquivos estão na  {\ttfamily pasta pacotes}. Ainda nesta mesma pasta está pacote \verb#subfigureppgmc# necessário para colocar figura lado a lado, conforme explicado na subseção \ref{figladoalado} e o arquivo {\ttfamily ppgmc-uesc} que contém toda a programação da classe. Esses arquivos não devem ser apagados ou alterados.

o comando \verb#\pretextual# indica o início da chamada dos elementos obrigatórios e opcionais. Os comandos  \verb#\imprimircapa# e \verb#\imprimirfolhaderosto{}# imprimem a capa e  a folha de rosto, respectivamente, com as informações contida no arquivo {\ttfamily 1-Inf.Capa-FolhaRosto} na pasta {\ttfamily 01-elementos-pre-textuais}. Tanto a capa quanto a folha de rosto são obrigatórias segundo as normas da ABNT e da UESC.

\textit{No arquivo \textit{{\ttfamily 1-Inf.Capa-FolhaRosto}} referido é que você dever colocar as informações principais do seu trabalho, tais como: título, autor, orientador, coorientador (caso não tenha coorientador comente a linha), instituição, tipo de trabalho (dissertação ou tese) etc. Tais informações aparecerão automaticamente na capa, folha de rosto, folha de aprovação.}

Este template leva em consideração que o autor possui coorientador, caso não possua basta comentar a linha 43 do arquivo \textit{{\ttfamily FolhaAprovacao}}, também na pasta {\ttfamily 01-elementos-pre-textuais}.


Na pasta {\ttfamily 01-elementos-pre-textuais} estão exemplos de arquivos pré-textuais, chamados no arquivo principal, a saber:
\begin{alineas}
\item \textbf{FolhaAprovacao}  -- elemento obrigatório
\item \textbf{dedicatoria} -- elemento opcional
\item \textbf{agradecimentos} -- elemento opcional
\item \textbf{epigrafe} -- elemento opcional
\item \textbf{resumoPt} -- elemento obrigatório
\item \textbf{resumoEn} -- elemento obrigatório
\item \textbf{listaFiguras} -- elemento opcional
\item \textbf{listaTabelas} -- elemento opcional 
\item \textbf{listaQuadros} -- elemento opcional 
\item \textbf{listaAlgoritmos} -- elemento opcional
\item \textbf{listaSiglas} -- elemento opcional
\item \textbf{listaSimbolos} -- elemento opcional
\item \textbf{sumário} -- elemento obrigatório
\end{alineas}

Os arquivos {\ttfamily listaFiguras, listaTabelas, listaQuadros e listaAlgoritmos} não precisam ser modificados. Caso algum destes elementos não seja necessário no seu trbalho, simplesmente comente no arquivo principal. Os arquivos  {\ttfamily listaSiglas e listaSimbolos} devem ser editados se fizerem parte do seu trabalho, caso contrário comente no arquivo principal.

A \textit{ficha catalográfica} é um elemento obrigatório fornecida pela biblioteca e deve ser colocado no verso da folha de rosto. No arquivo principal, a linha correspondente está comentada. Após ter em mão essa ficha, fornecida pela sua biblioteca, faça upload do arquvivo renomeado para \textit{{\ttfamily FichaCatalografica}} e descomente a linha correspondente no arquivo principal. 

O comando 	\verb#\textual# indica o início dos elementos textuais que são constituídos pela introdução, capítulos e conclusão. Já o comando \verb#\postextual# marca o início dos elementos pós-textuais, os quais são: referências, glossário, apêndices e índices. As referências devem ser armazenadas num arquivo de extensão \verb#.bib#. Na pasta \verb#03-elementos-pos-textuais# tem o arquivo {\ttfamily refbase.bib} que contém uma quantidade de referências como exemplo. Você pode substituir as que estão nesse arquivo pelas suas, sem necessidade de criar um novo arquivo. O capítulo \ref{chap:conclusao} traz mais detalhes.
\section{Coleta de dados}

Inserir seu texto aqui...

